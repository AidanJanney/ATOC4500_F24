\documentclass[11pt]{article} 
\usepackage[english]{babel}
\usepackage[utf8]{inputenc}
\usepackage[margin=0.5in]{geometry}
\usepackage{amsmath}
\usepackage{amsthm}
\usepackage{amsfonts}
\usepackage{amssymb}
\usepackage[usenames,dvipsnames]{xcolor}
\usepackage{graphicx}
\usepackage[colorinlistoftodos, color=orange!50]{todonotes}
\usepackage{hyperref}
\usepackage[numbers, square]{natbib}
\usepackage{fancybox}
\usepackage{epsfig}
\usepackage{soul}
\usepackage[framemethod=tikz]{mdframed}
\usepackage[shortlabels]{enumitem}
\usepackage[version=4]{mhchem}
\usepackage{multicol}
\usepackage{forest}
\usepackage{mathtools}
\usepackage{comment}
\usepackage{enumitem}
\usepackage[utf8]{inputenc}
\usepackage{listings}
\usepackage{color}
\usepackage[numbers]{natbib}
\usepackage{subfiles}
\usepackage{algorithm}
\usepackage[noend]{algpseudocode}
\usepackage{placeins}
\usepackage{booktabs}


\newtheorem{prop}{Proposition}[section]
\newtheorem{thm}{Theorem}[section]
\newtheorem{lemma}{Lemma}[section]
\newtheorem{cor}{Corollary}[prop]

\theoremstyle{definition}
\newtheorem{definition}{Definition}

\theoremstyle{definition}
\newtheorem{required}{Problem}
\newtheorem*{requiredHC}{Problem HC}

\theoremstyle{definition}
\newtheorem{ex}{Example}

\newcommand{\interval}[4]{\draw (#2, #1) -- (#3, #1); % Usage: \interval{height}{start}{end}{label}
\draw (#2, #1-0.11) -- (#2, #1+0.11); % draw left whisker
\draw (#3, #1-0.11) -- (#3, #1+0.11); % draw right whisker
\node[] at (#2-0.25, #1) {#4};
}


\setlength{\marginparwidth}{3.4cm}

\definecolor {processblue}{cmyk}{0.96,0,0,0}
\definecolor{processgreen}{rgb}{0, 255, 0}
%#########################################################

%To use symbols for footnotes
\renewcommand*{\thefootnote}{\fnsymbol{footnote}}
%To change footnotes back to numbers uncomment the following line
%\renewcommand*{\thefootnote}{\arabic{footnote}}

% Enable this command to adjust line spacing for inline math equations.
% \everymath{\displaystyle}

% _______ _____ _______ _      ______ 
%|__   __|_   _|__   __| |    |  ____|
%   | |    | |    | |  | |    | |__   
%   | |    | |    | |  | |    |  __|  
%   | |   _| |_   | |  | |____| |____ 
%   |_|  |_____|  |_|  |______|______|
%%%%%%%%%%%%%%%%%%%%%%%%%%%%%%%%%%%%%%%

\title{
\normalfont \normalsize 
\textsc{CSCI 3104 Fall 2024 \\ 
Instructor: Dr. Lijun Chen} \\
[10pt] 
\rule{\linewidth}{0.5pt} \\[6pt] 
\huge Problem Set 5 \\
\rule{\linewidth}{2pt}  \\[10pt]
}
%\author{}
\date{}

\begin{document}

\maketitle


%%%%%%%%%%%%%%%%%%%%%%%%%
%%%%%%%%%%%%%%%%%%%%%%%%%%
%%%%%%%%%%FILL IN YOUR NAME%%%%%%%
%%%%%%%%%%AND STUDENT ID%%%%%%%%
%%%%%%%%%%%%%%%%%%%%%%%%%%
\noindent
Due Date \dotfill October 08, 2024 (11:59 PM) \\
Name \dotfill \textbf{Aidan Janney} \\
Student ID \dotfill \textbf{109432417} \\
Collaborators \dotfill \textbf{Bodhi Rubinstein, Leo Nelson}

\section*{Instructions}
\addcontentsline{toc}{section}{Instructions}
 \begin{itemize}
	\item The solutions \textbf{should be typed}, using proper mathematical notation. We cannot accept hand-written solutions. \href{http://ece.uprm.edu/~caceros/latex/introduction.pdf}{Here's a short intro to \LaTeX.}
	\item You should submit your work through the \textbf{class Gradescope page} only (linked from Canvas). Please submit one PDF file, compiled using this \LaTeX \ template.
	\item You may not need a full page for your solutions; pagebreaks are there to help Gradescope automatically find where each problem is. Even if you do not attempt every problem, please submit this document with no fewer pages than the blank template (or Gradescope has issues with it).

	\item You are welcome and encouraged to collaborate with your classmates, as well as consult outside resources. You must \textbf{cite your sources in this document.} \textbf{Copying from any source is an Honor Code violation. Furthermore, all submissions must be in your own words and reflect your understanding of the material.} If there is any confusion about this policy, it is your responsibility to clarify before the due date. 

	\item Posting to \textbf{any} service including, but not limited to Chegg, Reddit, StackExchange, etc., for help on an assignment is a violation of the Honor Code.

	\item You \textbf{must} virtually sign the Honor Code (see Section \ref{HonorCode}). Failure to do so will result in your assignment not being graded.
\end{itemize}
\begin{table}[]
    \centering
    \begin{tabular}{lrrrrrrrrrrrr}
\toprule
Isotropic\_Error\_xyz & 0.0 & 0.1 & 0.2& 0.3 & 0.4 & 0.5 & 0.6 & 0.7 & 0.8 & 0.9 & 1.0 & Mean\_Difference \\
Initial\_Step &  &  &  &  &  &  &  &  &  &  &  &  \\
\midrule
18050 & 0.0 & 9.3 & 15.9 & 12.6 & 39.4 & 60.7 & 79.5 & 97.1 & 114.4 & 131.4 & 148.4 & 64.4 \\
7250 & 0.0 & 5.2 & 15.8 & 28.5 & 42.3 & 56.7 & 71.4 & 86.3 & 101.3 & 116.5 & 131.8 & 59.6 \\
18075 & 0.0 & 8.5 & 14.1 & 12.9 & 32.1 & 46.9 & 60.2 & 73.1 & 85.8 & 98.4 & 111.0 & 49.4 \\
16550 & 0.0 & 3.2 & 10.4 & 19.7 & 30.1 & 41.2 & 52.6 & 64.3 & 76.1 & 88.1 & 100.2 & 44.2 \\
16525 & 0.0 & 2.4 & 8.8 & 17.4 & 27.4 & 38.2 & 49.5 & 61.1 & 72.8 & 84.7 & 96.6 & 41.7 \\
7225 & 0.0 & 2.7 & 9.1 & 17.7 & 27.5 & 38.1 & 49.1 & 60.4 & 71.9 & 83.5 & 95.3 & 41.4 \\
7275 & 0.0 & 2.8 & 8.6 & 15.7 & 23.5 & 31.7 & 40.1 & 48.7 & 57.4 & 66.2 & 75.1 & 33.6 \\
11750 & 0.0 & 3.7 & 16.4 & 10.2 & 8.1 & 20.8 & 30.3 & 38.4 & 46.0 & 53.4 & 60.7 & 26.2 \\
18025 & 0.0 & 2.6 & 9.5 & 15.2 & 12.1 & 0.2 & 16.5 & 33.0 & 48.4 & 62.6 & 75.8 & 25.1 \\
11725 & 0.0 & 2.6 & 11.6 & 19.5 & 9.7 & 7.4 & 21.6 & 32.9 & 42.6 & 51.3 & 59.6 & 23.5 \\
\bottomrule
\end{tabular}
    \caption{Caption}
    \label{tab:my_label}
\end{table}

%%%%%%%%%%%%%%%%%%%%%%%%%%%%%%%%%%%%%%%%%%%%%%%%%%

\end{document} % NOTHING AFTER THIS LINE IS PART OF THE DOCUMENT



